\documentclass[10pt]{book}
\usepackage{fontspec}
\usepackage{microtype}
\usepackage[french]{babel}
\usepackage{csquotes}
\setmainfont{Old Standard}[RawFeature=onum]
\newfontfamily\hebfont{NewComputerModern10}[Script=Hebrew]
\usepackage[teiexport=tidy]{ekdosis}
\SetApparatus{
  delim={\unskip~‖ },
  direction=RL,
  sep={[ }
}
\SetHooks{
  lemmastyle=\hebfont,
  readingstyle=\hebfont
}
\def\LR#1{\bgroup\textdir TLT#1\egroup}
\DeclareWitness{B}{B}{Bodleian Pococke 296}
\DeclareWitness{V}{V}{Manuscrit Vatican 335}
\DeclareWitness{M}{M}{Moscou Guenzburg 341}[locus={61r-}]
\DeclareWitness{F}{F}{BNF Paris Ms. Héb. 982}
\DeclareWitness{C}{C}{Bodleian Mich. 413}
\DeclareWitness{N}{N}{New York JTSA 2497}
\DeclareWitness{A}{A}{Milan Ambrosiana Ms X 132 Sup.}
\DeclareWitness{D}{D}{Bodleian Mich. 413}
\DeclareWitness{G}{G}{Moscou Guenzburg 1184}
\DeclareWitness{H}{H}{Harvard Ms. Heb. 58}
\DeclareWitness{E}{E}{Munich Cod. Heb. 33}
\DeclareWitness{P}{P}{Parma Palatina 2449}
\DeclareWitness{F}{F}{New York JTSA Ms. 2415}
\DeclareWitness{BL}{\LR{BL}}{British Library Or. 1084}

\def\ppar{\par}
\FormatDiv{1}{\noindent\bfseries}{\ppar}

\AtBeginEnvironment{hebrew}{\setRL\hebfont}

\usepackage{xltabular}
\usepackage{nextpage}

\begin{document}
\begin{xltabular}[c]{.7\linewidth}{lXl}
  \caption*{\bfseries Conspectus Siglorum}\\
  \SigLine{B}\\
  \SigLine{V}\\
  \SigLine{M}\\
  \SigLine{F}\\
  \SigLine{C}\\
  \SigLine{N}\\
  \SigLine{A}\\
  \SigLine{D}\\
  \SigLine{G}\\
  \SigLine{H}\\
  \SigLine{E}\\
  \SigLine{P}\\
  \SigLine{F}\\
  \SigLine{BL}\\
\end{xltabular}

\SetAlignment{
  tcols=2,
  lcols=1,
  texts=french;hebrew,
  apparatus=hebrew
}

\cleartoevenpage

\begin{alignment}
  \begin{french}
    \ekddiv{head=Premier chapitre, n=1}

    Sur le secret des six questions et des six réponses qui étaient
    demandées à propos des secrets des \enquote{Ourim et Toumim},
    \enquote{et ils se termineront en jointure […] dans le premier
      anneau} (Exode 26, 24). Et elles sont celles qui sont indiquées
    dans les douze maisons du Nom honoré de quatre lettres, et leurs
    formes changeantes, et les quatre périodes exemplifiées dans le
    pectoral [ḥošen, un des vêtements du grand prêtre] et leurs
    points. Et quelques-uns des anciens qui cherchaient à trouver les
    paroles de volonté ont référé la compréhension de ce secret à
    quelques traces, et ce sont les versets \enquote{des fils
      d’Issacar, qui savaient discerner les temps afin de connaître ce
      que devait faire Israël} (I Chr. 12, 33), et \enquote{un homme
      tout prêt} (Lev 16,21), et c’est un secret extrêmement caché
    dont je donnerai quelques pistes dans ce qui suit.
  \end{french}
  
  \begin{hebrew}
  \ekddiv{head={הפרק הראשון}, n=1}

 בסוד שש שאלות \app{\lem{ושש}}{
      \rdg[wit=N]{ושלש}
    } \app{\lem[sep={ +}]{תשובות}}{
      \rdg[wit={G, N, A, C, V, BL, B, M}]{הנשאלות}
    } \app{\lem{בסוד}}{
      \rdg[wit={BL}]{בסודי}
    } אורים \app{\lem{ותומים}}{
      \rdg[wit={G}]{ותומ}
    } \emph{\app{\lem{ויחדו}}{
      \rdg[wit={G, N, C, F, M}]{ויחדיו}
    } יהיו תמים אל הטבעת
\app{\lem{האחת}}{
  \rdg[wit={M}]{הזאת}
}} והם המכוונות בסוד שנים עשר בתי השם הנכבד \app{\lem{בן}}{
\rdg[wit={BL}]{בין}
} ארבע אותיות
ובצורותיהם \app{\lem{המשתנות}}{
  \rdg[wit={BL}]{המשתיות}
} ותקופותים הארבע הנמשלות בטורי החשן ובנקודיהם. ומקצת
הקדמונים המבקשים למצוא דברי חפץ נתיחסו הבנת קצת רושם זה הסוד והם שנאמ'
בערם ומבני יששכר יודעי בינה לעתים לדעת מה יעשה ישראל ונאמר גם כן איש
עתי והוא סוד נעלם בתכלית העלמה ארמוז אליו קצת רמז תכף זה.

\end{hebrew}
\end{alignment}
\end{document}
